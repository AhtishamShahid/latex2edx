\documentclass[12pt]{article}
\usepackage{amsmath}

\usepackage{graphicx}
\usepackage{edXpsl}	% edX 
%%%%%%%%%%%%%%%%%%%%%%%%%%%%%%%%%%%%%%%

\begin{document}

\begin{edXcourse}{8.02x}{Electricity \& Magnetism}

\begin{edXchapter}{Week9}

\begin{edXsequential}

\begin{edXproblem}{ps2_p5}{display_name="Two spherical conductors" showanswer="always" rerandomize="never" weight="15"}


\begin{edXscript}
import math
VA=80
VB=-30
e=1.602e-19 
me=9.1e-31 
mp=1.672e-27 
eps0=8.854e-12
mu0=4e-7*math.pi 
parta=math.sqrt(2*e*(VA-VB)/(me))
\end{edXscript}

%\begin{center}
%\includegraphics[width=4in]{images/ps2_p5_fig1.png}
%\end{center}


There are two spherical conductors, $A$ and $B$. They are placed in vacuum. $A$ has a 
radius $r_A=25$ cm and $B$ of $r_B=35$ cm. The distance between the centers of the two spheres 
is $d=225$ cm.

$A$ has a potential of $V_A=+80$ Volt and $B$ has a potential of $V_B=-30$ Volt.

a. An electron is released with zero speed from $B$. What will its speed be as it 
reaches $A$? Express you answer in terms of the following variables,
if relevant, $r_A$, $r_B$,$V_A$, $V_B$, $d$, $m_e$, $m_p$ and natural
constants. 

Try script value: \$parta
Try using script value in answer
%%%%%%%%%%%%%%%%%%%%%%%%%%%%%%%%%%%
%%% NOT WORKING
\edXabox{type="numerical" expect="\$parta" tolerance="5\%"}
%%%%%%%%%%%%%%%%%%%%%%%%%%%%%%%%%%%%%
%\edXabox{type="symbolic" expect="sqrt(2*e*(V_A-V_B)/(m_e))"} 


b. A proton is released with zero speed from $A$. What will its speed be as it reaches 
$B$? Express you answer in terms of the following variables,
if relevant, $r_A$, $r_B$,$V_A$, $V_B$, $d$, $m_e$, $m_p$ and natural constants.

\edXabox{type="symbolic" expect="sqrt(2*e*(V_A-V_B)/(m_p))"} 

c. We now change only the potential of $B$ to $V_B=+25$ Volt. What now is the ratio  

of the speed of the electron (as it arrives at $A$) and the speed of the  


proton (as it arrives at $B$)? Express you answer in terms of the following variables,
if relevant, $r_A$, $r_B$,$V_A$, $V_B$, $d$, $m_e$, $m_p$ and natural
constants.

\edXabox{type="symbolic" expect="sqrt(m_p/m_e)"} 

\begin{edXsolution}
The electron has charge $q_{e} =-e $ \textbf{ }and mass $m_{e}  $ .\textbf{ }Let 
 $\Delta V_{BA} =V_{A} -V_{B}  $  denote the electric potential
 difference between the 
spheres. The change in potential energy $\Delta U_{BA} =U_{A} -U_{B}  $  
when the electron is moved from the initial position is

$$\Delta U_{BA} =q\Delta V_{BA} =-e(V_{A} -V_{B} )<0 $$

The electron was released from rest, therefore the change in kinetic energy is 

$$\Delta K_{BA} \equiv K_{A} -K_{B} =\frac{1}{2} m_{e} v_{A}^{2} -\frac{1}{2} m_{e} v_{B}^{2} =\frac{1}{2} m_{e} v_{A}^{2}  $$
 

From conservation of energy, 

$$0=\Delta K_{BA} +\Delta U_{BA} =\frac{1}{2} m_{e} v_{A}^{2} -e(V_{A} -V_{B} ) $$

So the magnitude of the velocity of the electron when it reaches the
surface of $A$
is, 

$$v_{A} =\sqrt{\frac{2e(V_{A} -V_{B} )}{m_{e} } }  $$

b.)

The proton has $q_{p} =e $ \textbf{, }and mass\textbf{ $m_{p}  $ }. The change 
in potential energy $\Delta U_{AB} =U_{B} -U_{A}  $  when the proton is moved from 
the initial position   to the final position  is, 

$$\Delta U_{AB} =q_{p} \Delta V_{AB} =e(V_{B} -V_{A} )<0 $$

The proton was released from rest, therefore the change in kinetic energy is 

$$\Delta K_{AB} \equiv K_{B} -K_{A} =\frac{1}{2} m_{p} v_{B}^{2} -\frac{1}{2} m_{p} v_{A}^{2} =\frac{1}{2} m_{p} v_{B}^{2}  $$
 

From conservation of energy, 

$$0=\Delta K_{AB} +\Delta U_{AB} =\frac{1}{2} m_{p} v_{B}^{2} +e(V_{B} -V_{A} ) $$

So the magnitude of the velocity of the proton when it reaches the
surface of sphere $B$
is, 

$$v_{B} =\sqrt{-\frac{2e(V_{B} -V_{A} )}{m_{p} } }  $$

c.) Using our results above we have that

$$\frac{v_{A} }{v_{B} } =\frac{\sqrt{\frac{2e(V_{A} -V_{B} )}{m_{e} } } }{\sqrt{-\frac{2e(V_{B} -V_{A} )}{m_{p} } } } =\sqrt{\frac{m_{p} }{m_{e} } }  $$


The ratio is independent of the potential difference.
\end{edXsolution}

\end{edXproblem}

%%%%%%%%%%%%%%%%%%%%%%%%%%%%%%%%%%%%%%%%
\end{edXsequential}
\end{edXchapter}
\end{edXcourse}

\end{document}
