\documentclass[12pt]{article}

\usepackage{edXpsl}	% edX

%%%%%%%%%%%%%%%%%%%%%%%%%%%%%%%%%%%%%%%%%%%%%%%%%%%%%%%%%%%%%%%%%%%%%%%%%%%%%
\begin{document}

\begin{edXproblem}{p1}{display_name="Example formularesponse problem written in latex"}

\section{Question 2}

What is Einstein's equation for the relativistic energy of a mass $m$?

%        <abox type="formula" expect="m*c^2" samples="m,c@1,2:3,4#10" type="cs" size="40" math="1" tolerance="0.01" />

%        format of samples:  <variables>@<lower_bounds>:<upper_bound>#<num_samples
%
%        * variables    - a set of variables that are allowed as student input
%        * lower_bounds - for every variable defined in variables, a lower
%                         bound on the numerical tests to use for that variable
%        * upper_bounds - for every variable defined in variables, an upper
%                         bound on the numerical tests to use for that variable

\edXabox{type="formula" expect="m*c^2" math="1" samples="m,c@1,-10:100,10#32" math="1" }


\end{edXproblem}

%%%%%%%%%%%%%%%%%%%%%%%%%%%%%%%%%%%%%%%%

\end{document}
